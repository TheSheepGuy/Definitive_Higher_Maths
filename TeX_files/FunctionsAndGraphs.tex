% Define number set symbols.
\newcommand{\N}{\mathbb{N}}
\newcommand{\Z}{\mathbb{Z}}
\newcommand{\Q}{\mathbb{Q}}
\newcommand{\R}{\mathbb{R}}
\newcommand{\C}{\mathbb{C}}

\chapter{Functions and Graphs}
\section{Composite Functions}
A function $f(x) = 3x+9$ will take in any value for $x$, and substitutes it into the expression $3x+9$. If another expression would be passed into $f(x)$, such as x-2, then the result would be $3(x-2)+9$. A composite function is one where two functions are combined, similar to above.

Suppose $f(x) = 2x$ and $g(x) = x+1$, then composite functions $f(g(x))$ and $g(f(x))$ can be found as follows:
\begin{align*}
	f(g(x)) &= 2(x+1) & g(f(x)) &= (2x)+1\\
	&=2x+2 & &=2x+1
\end{align*}

A composite function can get tiring to write out all the time. So it could be written as $h(x)=f(g(x))$.

\subsection{Example}
$f(x)=x^2+1$ and $g(x)=\frac{1}{x} \left(x\neq0\right)$. Find $h(x)=f(g(x))$ and $h(x)=g(f(x))$.
\begin{align*}
	h(x)&=f\left(\frac{1}{x}\right) & k(x)&=g\left(x^2+1\right)\\
	&=\left(\frac{1}{x}\right)^2-2 & &=\frac{1}{x^2+1}\\
	&=\frac{1}{x^2}-1
\end{align*}


\section{Domains and Set Notation}
A function most often can't take all numbers as inputs. For example, a function of the area of a rectangle might be written as $f(x)=x^2+2x-8$. Since it's a function of real-life area in terms of lengths, the area cannot be negative. After some thinking, it can be concluded that $x$ must be 2 or larger. So the function's domain is any number larger or equal to 2.

Additionally, a real number (aka decimal number) could also be passed in to the function, but maybe that's not what you want. You can also define the domain to only include natural numbers (positive whole numbers).

Applying these two factors, the domain of the function is $\{x \in \N \mid x \geq 2\}$

Set (or rather set-builder) notation is a way of describing a set of numbers. In relation to domains, it details what set of numbers can be an acceptable input of the function. The kinds of set number symbols that exist commonly are
\begin{itemize}
	\item $\N$, natural numbers. Non-negative integers (so starting with 0, counting up in steps of 1).
	\item $\Z$, integers. Any number that can be written without a fractional component (``whole" numbers).
	\item $\Q$, rational numbers. A number which can be written as a fraction with an integer numerator and a non-zero natural denominator\footnote{Negative denominators can exist, but are avoided, as they can be expressed as a negative numerator instead.}.
	\item $\R$, real numbers. Any number that can be represented on a number line.
\end{itemize}

\subsection{Example}
Find a suitable domain for the function $\frac{1}{x^2-1}$.

\begin{equation*}
	\{x \in \R \mid x \neq \sqrt{2}\}
\end{equation*}